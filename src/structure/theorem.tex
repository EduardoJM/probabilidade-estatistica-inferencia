% For math equations, theorems, symbols, etc
\usepackage{amsmath, amsfonts, amssymb, amsthm}

\newcommand{\intoo}[2]{\mathopen{]}#1\,;#2\mathclose{[}}
\newcommand{\ud}{\mathop{\mathrm{{}d}}\mathopen{}}
\newcommand{\intff}[2]{\mathopen{[}#1\,;#2\mathclose{]}}
\renewcommand{\qedsymbol}{$\blacksquare$}
\newtheorem{notation}{Notation}[chapter]

% Boxed/framed environments
\newtheoremstyle{ocrenumbox} % Theorem style name
	{0pt} % Space above
	{0pt} % Space below
	{\normalfont} % Body font
	{} % Indent amount
	{\small\bf\sffamily\color{ocre}} % Theorem head font
	{\;}% Punctuation after theorem head
	{0.25em}% Space after theorem head
	{
		% Theorem text (e.g. Theorem 2.1)
		\small\sffamily\color{ocre}\thmname{#1}\nobreakspace\thmnumber{\@ifnotempty{#1}{}\@upn{#2}}
		% Optional theorem note
		\thmnote{\nobreakspace\the\thm@notefont\sffamily\bfseries\color{black}---\nobreakspace#3.}
	}

\newtheoremstyle{blacknumex} % Theorem style name
	{5pt} % Space above
	{5pt} % Space below
	{\normalfont} % Body font
	{} % Indent amount
	{\small\bf\sffamily} % Theorem head font
	{\;} % Punctuation after theorem head
	{0.25em}% Space after theorem head
	{
		% Theorem text (e.g. Theorem 2.1)
		\small\sffamily{\tiny\ensuremath{\blacksquare}}\nobreakspace\thmname{#1}\nobreakspace\thmnumber{\@ifnotempty{#1}{}\@upn{#2}}
		% Optional theorem note
		\thmnote{\nobreakspace\the\thm@notefont\sffamily\bfseries---\nobreakspace#3.}
	}

\newtheoremstyle{blacknumbox} % Theorem style name
	{0pt} % Space above
	{0pt} % Space below
	{\normalfont} % Body font
	{} % Indent amount
	{\small\bf\sffamily} % Theorem head font
	{\;} % Punctuation after theorem head
	{0.25em} % Space after theorem head
	{
		% Theorem text (e.g. Theorem 2.1)
		\small\sffamily\thmname{#1}\nobreakspace\thmnumber{\@ifnotempty{#1}{}\@upn{#2}}
		% Optional theorem note
		\thmnote{\nobreakspace\the\thm@notefont\sffamily\bfseries---\nobreakspace#3.}
	}

% Non-boxed/non-framed environments
\newtheoremstyle{ocrenum} % Theorem style name
	{5pt} % Space above
	{5pt} % Space below
	{\normalfont} % Body font
	{} % Indent amount
	{\small\bf\sffamily\color{ocre}} % Theorem head font
	{\;} % Punctuation after theorem head
	{0.25em} % Space after theorem head
	{
		% Theorem text (e.g. Theorem 2.1)
		\small\sffamily\color{ocre}\thmname{#1}\nobreakspace\thmnumber{\@ifnotempty{#1}{}\@upn{#2}}
		% Optional theorem note
		\thmnote{\nobreakspace\the\thm@notefont\sffamily\bfseries\color{black}---\nobreakspace#3.}
	}

\makeatother

% Defines the theorem text style for each type of theorem to one of the three styles above
\newcounter{dummy}
\numberwithin{dummy}{chapter}
\theoremstyle{ocrenumbox}
\newtheorem{theoremeT}[dummy]{Teorema}
\newtheorem{problem}{Problema}[chapter]
\newtheorem{exerciseT}{Exercícios}[chapter]
%\theoremstyle{blacknumex}
\newtheorem{exampleT}{Exemplo}[chapter]
\theoremstyle{blacknumbox}
\newtheorem{vocabulary}{Vocabulary}[chapter]
\newtheorem{definitionT}{Definição}[chapter]
\newtheorem{corollaryT}[dummy]{Corolário}
\theoremstyle{ocrenum}
\newtheorem{proposition}[dummy]{Proposição}

% Required for creating the theorem, definition, exercise and corollary boxes
\RequirePackage[framemethod=default]{mdframed}
% Theorem box
\newmdenv[
	skipabove=7pt,
	skipbelow=7pt,
	backgroundcolor=black!5,
	linecolor=ocre,
	innerleftmargin=5pt,
	innerrightmargin=5pt,
	innertopmargin=5pt,
	leftmargin=0cm,
	rightmargin=0cm,
	innerbottommargin=5pt
]{tBox}

% Exercise box	  
\newmdenv[
	skipabove=7pt,
	skipbelow=7pt,
	rightline=false,
	leftline=true,
	topline=false,
	bottomline=false,
	backgroundcolor=ocre!10,
	linecolor=ocre,
	innerleftmargin=5pt,
	innerrightmargin=5pt,
	innertopmargin=5pt,
	innerbottommargin=5pt,
	leftmargin=0cm,
	rightmargin=0cm,
	linewidth=4pt
]{eBox}

% Example box	  
\newmdenv[
	skipabove=7pt,
	skipbelow=7pt,
	rightline=false,
	leftline=true,
	topline=false,
	bottomline=false,
	backgroundcolor=ocre!10,
	linecolor=ocre,
	innerleftmargin=5pt,
	innerrightmargin=5pt,
	innertopmargin=5pt,
	innerbottommargin=5pt,
	leftmargin=0cm,
	rightmargin=0cm,
	linewidth=4pt
]{exampleBox}	

% Definition box
\newmdenv[
	skipabove=7pt,
	skipbelow=7pt,
	rightline=false,
	leftline=true,
	topline=false,
	bottomline=false,
	linecolor=ocre,
	innerleftmargin=5pt,
	innerrightmargin=5pt,
	innertopmargin=0pt,
	leftmargin=0cm,
	rightmargin=0cm,
	linewidth=4pt,
	innerbottommargin=0pt
]{dBox}	

% Corollary box
\newmdenv[
	skipabove=7pt,
	skipbelow=7pt,
	rightline=false,
	leftline=true,
	topline=false,
	bottomline=false,
	linecolor=gray,
	backgroundcolor=black!5,
	innerleftmargin=5pt,
	innerrightmargin=5pt,
	innertopmargin=5pt,
	leftmargin=0cm,
	rightmargin=0cm,
	linewidth=4pt,
	innerbottommargin=5pt
]{cBox}

% Remark box	  
\newmdenv[
	skipabove=7pt,
	skipbelow=7pt,
	rightline=false,
	leftline=false,
	topline=true,
	bottomline=true,
	backgroundcolor=ocre!10,
	linecolor=ocre,
	innerleftmargin=5pt,
	innerrightmargin=5pt,
	innertopmargin=5pt,
	innerbottommargin=5pt,
	leftmargin=0cm,
	rightmargin=0cm,
	linewidth=8pt
]{remBox}

% Theorem Environment
\newenvironment{theorem}{
	\begin{tBox}\begin{theoremeT}
}{
	\end{theoremeT}\end{tBox}
}
% Exercise Environment
\newenvironment{exercise}{
	\begin{eBox}\begin{exerciseT}
}{
	\end{exerciseT}\end{eBox}
}
% Definition Environment
\newenvironment{definition}{
	\begin{dBox}\begin{definitionT}
}{
	\end{definitionT}\end{dBox}
}	
% Example Environment
\newenvironment{example}{
	\begin{exampleBox}\begin{exampleT}
}{
	\hfill{\color{ocre}\tiny\ensuremath{\blacksquare}}\end{exampleT}\end{exampleBox}
}
% Corollary Environment
\newenvironment{corollary}{
	\begin{cBox}\begin{corollaryT}
}{
	\end{corollaryT}\end{cBox}
}
% Note Environment
\newenvironment{note}[1]{
	\begin{remBox}
	\textbf{#1}\par\vspace{10pt}
}
{
	\end{remBox}
}

% Remark Environment
\newenvironment{remark}{
	\par\vspace{10pt}\small % Vertical white space above the remark and smaller font size
	\begin{list}{}{
		\leftmargin=35pt % Indentation on the left
		\rightmargin=25pt
	}
		\item
			\ignorespaces % Indentation on the right
			\makebox[-2.5pt]{
				\begin{tikzpicture}[overlay]
					\node[
						draw=ocre!60,
						line width=1pt,
						circle,
						fill=ocre!25,
						font=\sffamily\bfseries,
						inner sep=2pt,
						outer sep=0pt
					] at (-15pt,0pt){\textcolor{ocre}{R}};
				\end{tikzpicture}
			}
			\advance\baselineskip -1pt
}{
	\end{list}\vskip5pt
}